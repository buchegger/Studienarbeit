
\usepackage{ae}

% Wir verwenden die neue deutsche Rechtschreibung (wir versuchen es zumindest ^^)
\usepackage{ngerman}
\usepackage[english,ngerman]{babel}

% Wir m�hten Grafiken verwenden und diese sollen als Ausgabetreiber pdfTex verwenden
\usepackage[pdftex]{graphicx}

% Keine Papierverschwendung wie es bei Tex-Standard blich ist
\usepackage[bottom=30mm,top=25mm,inner=20mm,outer=20mm,marginparwidth=15mm,marginparsep=3mm,headsep=10mm]{geometry}

% Die zus�zlichen AMS-Mathepakete, und sch�ere Integralgrenzen
\usepackage[intlimits]{amsmath}
\usepackage{amsthm}
\usepackage{amsopn}
\usepackage{amscd}
\usepackage{amsfonts}
\usepackage{amssymb}
%\usepackage{mathrsfs}

% Einheitenbefehle \unit und \unitfrac
\usepackage{units}

% Sch�ere und flexiblere Kopfzeilen
\usepackage{fancyhdr}

% Ab und zu m�hten wir Seiten im Querformat einbauen
\usepackage{pdflscape}

% Vorschau PDFs erstellen
\usepackage{thumbpdf}

% Type1-Fonts (damit die PDFs nicht so verpixeln)
% Folgende Pakete sollten deshalb installiert sein:
% Font-Packages: CTAN: fonts/cmbright/, CTAN: fonts/ps-type1/cm-super/, CTAN: fonts/ps-type1/hfbright/
\usepackage{type1cm}

% Komplette PDFs (oder Seiten daraus) importieren
\usepackage{pdfpages}

% If-Abfragen erm�lichen
\usepackage{ifthen}

% Quellcodes sch� darstellen
\usepackage{verbatim}

% Erweitere Quellcodedarstellung
\usepackage{listings}

% Verschiedene Pakete (was machten die nochmals?)
\usepackage{titling}
\usepackage{textcomp}
\usepackage{nonfloat}
\usepackage{booktabs}

% Erweitere Aufz�lungen
\usepackage{eqlist}
\usepackage{paralist}

% Farbuntersttzung
\usepackage{color}

% PDF spezifische Funktionen
\usepackage[a4paper,breaklinks,unicode,colorlinks,linktocpage,pdftex,linkcolor=blue,urlcolor=blue,backref,pagebackref,bookmarks,bookmarksnumbered]{hyperref}

% dsfont
\usepackage{multicol,wasysym,expdlist}

% kleine Layoutfehler fixen
\usepackage{ellipsis,fixltx2e,mparhack}

\usepackage{microtype}

% absolute Textpositionierung
\usepackage{textpos}

% Indexerstellung
\usepackage{makeidx}
\usepackage{booktabs}

% Immer Serifenlos schreiben
\renewcommand{\familydefault}{\sfdefault}

% Abs�ze nicht einrcken
\setlength{\parindent}{0em}

% Titel, Autor, Datum konfigurieren (fr Titelseite)
\title{ \textbf{\Titel} \\ \Thema}
\ifthenelse{ \equal{\Gruppe}{} }{ \author{\Autoren} }{ \author{\Autoren \\ \Gruppe}}
\date{\Datum}

% PDF Informationen
\hypersetup{
  pdftitle=\Titel,
  pdfsubject=\Thema,
  pdfauthor=\Autoren,
  pdfkeywords=\Schluesselwoerter,
  pdfcreator={LaTeX},
  pdfproducer={LaTeX}
  %pdfpagemode=None
  %pdfpagelayout=Singlepage  (keine Lesezeichen)
}

\fancypagestyle{plain}{%
}

% C++ Layout fuer Quellcodes importieren
\lstloadlanguages{c++}

\begin{document}
\definecolor{Gray}{gray}{0.5}

% Farben und Quellcodedefinitionen fr die sch�e Darstellung von LISP/Scheme
\definecolor{darkblue}{rgb}{0,0,.6}
\definecolor{darkred}{rgb}{.6,0,0}
\definecolor{darkgreen}{rgb}{0,.6,0}
\definecolor{red}{rgb}{.98,0,0}
%
\lstset{
  numbers=left,
  numberblanklines=false,
  showspaces=false,
  showstringspaces=false,
  numberstyle=\tiny,
  language=C++,
  inputencoding=utf8,
  extendedchars=true,
  basicstyle=\ttfamily,
  commentstyle=\itshape\color{Gray},
  keywordstyle=\bfseries\color{darkblue},
  directivestyle=\color{darkgreen},
  stringstyle=\color{darkred},
  breaklines,
  postbreak=\space,
  breakindent=5pt,
  tabsize=8}

% normalerweise mit Buchstaben a), b) aufz�len
\renewcommand{\labelenumi}{\alph{enumi})}

% keine Nummerierung der �erschriften
\setcounter{secnumdepth}{0}

% etwas kleinere Formelabst�de
\setlength{\jot}{6pt}

% Seitenlayout mit Kopfzeilen definieren
\pagestyle{fancy}
\rhead{\today}
\chead{}
\lhead{\textbf{\Titel} \\ \Thema}
\lfoot{\Autoren}
\cfoot{}
\rfoot{\thepage}
\renewcommand{\headrulewidth}{0.4pt}
\renewcommand{\footrulewidth}{0.4pt}
\renewcommand{\headheight}{25.pt}

% Titelseite erstellen
\maketitle

% Und etwas Platz darunter lassen
\vspace{1em}